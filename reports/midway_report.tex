\documentclass{article} % For LaTeX2e
\usepackage{nips15submit_e,times}
\usepackage{hyperref}
\usepackage{url}
%\documentstyle[nips14submit_09,times,art10]{article} % For LaTeX 2.09


\title{Applying Convolutional Neural Networks (CNNs) for Decoding fMRI Images}


\author{
David S.~Hippocampus\thanks{ Use footnote for providing further information
about author (webpage, alternative address)---\emph{not} for acknowledging
funding agencies.} \\
Department of Computer Science\\
Cranberry-Lemon University\\
Pittsburgh, PA 15213 \\
\texttt{hippo@cs.cranberry-lemon.edu} \\
\And
Coauthor \\
Affiliation \\
Address \\
\texttt{email} \\
\AND
Coauthor \\
Affiliation \\
Address \\
\texttt{email} \\
\And
Coauthor \\
Affiliation \\
Address \\
\texttt{email} \\
\And
Coauthor \\
Affiliation \\
Address \\
\texttt{email} \\
(if needed)\\
}

% The \author macro works with any number of authors. There are two commands
% used to separate the names and addresses of multiple authors: \And and \AND.
%
% Using \And between authors leaves it to \LaTeX{} to determine where to break
% the lines. Using \AND forces a linebreak at that point. So, if \LaTeX{}
% puts 3 of 4 authors names on the first line, and the last on the second
% line, try using \AND instead of \And before the third author name.

\newcommand{\fix}{\marginpar{FIX}}
\newcommand{\new}{\marginpar{NEW}}

%\nipsfinalcopy % Uncomment for camera-ready version

\begin{document}


\maketitle

\begin{abstract}
We did what on which data set (using CNNs to decode 3-D image and predict word based on brain image). Our network looks like what. Some additional tricks we used are what. Our performance was what (comparing with baseline logistic regression classifier).
\end{abstract}

\section{Introduction}

People tried stuff. Some current techniques are what.
To do something, we did what. In order to do so, this stuff and that stuff needed to be done.
The specific contributions of this paper are as follows.
In the end, the project suggests what.

\section{Dataset}

Talk about the dataset.

\section{Baseline Approach}

Talk about it.

\section{CNN Architecture}

Talk about it.

\subsection{Layers Used}

Talk about it.

\subsubsection{Layer 1}

Talk about it.

\subsubsection{Layer 2}

Talk about it.

\subsection{Units Used}

Talk about it.

\subsubsection{Unit 1}

Talk about it.

\subsection{Other Techniques or Stuff}

\section{Dealing with 3D}

Talk about it.

\section{Results}

Show table. Compare accuracies based on different settings. Compare with baseline and random choice.

\subsection{Qualitative Evaluation}

Maybe talk about sample images.

\section{Discussion}

Our results show something. Some possible improvements might be something.

\section*{References}

\small{
[1] Alexander, J.A. \& Mozer, M.C. (1995) Template-based algorithms
for connectionist rule extraction. In G. Tesauro, D. S. Touretzky
and T.K. Leen (eds.), {\it Advances in Neural Information Processing
Systems 7}, pp. 609-616. Cambridge, MA: MIT Press.

[2] Bower, J.M. \& Beeman, D. (1995) {\it The Book of GENESIS: Exploring
Realistic Neural Models with the GEneral NEural SImulation System.}
New York: TELOS/Springer-Verlag.

[3] Hasselmo, M.E., Schnell, E. \& Barkai, E. (1995) Dynamics of learning
and recall at excitatory recurrent synapses and cholinergic modulation
in rat hippocampal region CA3. {\it Journal of Neuroscience}
{\bf 15}(7):5249-5262.
}

\end{document}